\documentclass[../main.tex]{subfiles}
 
\newcommand{\set}[1]{\mathcal {#1}}
\newcommand{\distribution}[1]{\mathcal {#1}}
\newcommand{\sized}[1]{\tilde \set {#1}}

\newcommand{\radius}{r}
\newcommand{\dist}[2]{\distf({#1},{#2})}
\newcommand{\distf}{d}
\newcommand{\diam}[1]{\text{diam}({#1})}
\newcommand{\codiam}[1]{\text{codiam}({#1})}
\newcommand{\aspect}[1]{\Delta({#1})}

\newcommand{\minkdim}{\text{dim}_\text{Mink}}
\newcommand{\krdim}{\text{dim}_\text{kr}}
\newcommand{\doubdim}{\text{dim}_\text{doub}}

\newcommand{\krnum}{c_\text{kr}}
\newcommand{\doubnum}{c_\text{doub}}

\newcommand{\p}{\ensuremath p}
\newcommand{\q}{\ensuremath q}
%\newcommand{\varfont}[1]{\ensuremath{\textup{\text{{#1}}}}}
\newcommand{\mkfunction}[1]{\ensuremath{\text{{#1}}}}
\newcommand{\level}[1]      {\mkfunction{level}({#1})}
\newcommand{\parent}[1]     {\mkfunction{parent}({#1})}
\newcommand{\children}[1]   {\mkfunction{children}({#1})}
\newcommand{\covdist}[1]    {\mkfunction{covdist}({#1})}
\newcommand{\descendants}[1]{\mkfunction{descendants}({#1})}
\newcommand{\maxdist}[1]    {\mkfunction{maxdist}({#1})}
\newcommand{\height}[1]     {\mkfunction{height}({#1})}

% FIXME: should these be changed?
\newcommand{\nn}[1]{\ensuremath{\ensuremath{{{#1}}_{nn}}}}
\newcommand{\exprad}[1]{\ensuremath{\ensuremath{2}}}
\newcommand{\pack}{\ensuremath{\text{\ttfamily pack}}}
\newcommand{\rmNodes}{\ensuremath{\text{\ttfamily rmNodes}}}
\newcommand{\findnn}{\ensuremath{\text{\ttfamily findNearestNeighbor}}}
\newcommand{\ctmerge}{\ensuremath{\text{\ttfamily merge}}}
\newcommand{\ctinsert}{\ensuremath{\textnormal{\ttfamily insert}}}
\newcommand{\ctinsertHelper}{\ensuremath{\text{\ttfamily insert\_}}}
\newcommand{\rebalance}{\ensuremath{\text{\ttfamily rebalance}}}
\newcommand{\rebalanceHelper}{\ensuremath{\text{\ttfamily rebalance\_}}}
\newcommand{\mkvar}[1]{\ensuremath{\text{\emph{{#1}}}}}
\newcommand{\nullvar}{\ensuremath{\textup{\text{\ttfamily null}}}}
\newcommand{\datapoint}[1]{\ensuremath{\textup{\text{\ttfamily dp}({#1})}}}

%%%%%%%%%%%%%%%%%%%%%%%%%%%%%%%%%%%%%%%%%%%%%%%%%%%%%%%%%%%%%%%%%%%%%%%%%%%%%%%%

\begin{document}
\chapter{Cover Trees}

\begin{definition}
    A set $\set X$ equipped with a distance function $\distf : \set X \times \set X \to \R$ is a \emph{metric space} if it obeys the following properties:
    \begin{enumerate}
        %\item \emph{Non-negativity}.  For all $x_1,x_2\in\set X$, $\dist{x_1}{x_2} \ge 0$.
        \item \emph{Indiscernability}.  For all $x_1,x_2\in\set X$, $\dist{x_1}{x_2} = 0$ if and only if $x_1=x_2$.
        \item \emph{Symmetry}. For all $x_1,x_2\in\set X$, $\dist{x_1}{x_2} = \dist{x_2}{x_2}$.
        \item \emph{Triangle inequality}.  For all $x_1,x_2,x_3\in\set X$, $\dist{x_1}{x_2} + \dist{x_2}{x_3}\ge\dist{x_1}{x_3}$.
    \end{enumerate}
\end{definition}

\begin{table}[H]
    \small
    %\newcolumntype{Y}{>{\centering\arraybackslash}p{1in}}
    %\newcolumntype{Y}{p{0.1in}}
    %\begin{tabularx}{\textwidth}{lXXY}
    \centering
    \begin{tabular}{lccc}
        \toprule
        \vspace{-0.25in}
        &~\hspace{1.2in}~&~\hspace{1.2in}~&~\hspace{1.2in}~\\
        data structure & space & find nearest neighbor & insertion \\
        \midrule
        ball tree \cite{} & $O(n)$ & $O(n)$ & $O(n)$ \\
        metric skip list \cite{karger2002finding} & $O(n\log n)$ & $\krnum{}^{O(1)}\log n$ & $\krnum^{O(1)}\log n\log\log n$ \\
        navigating net \cite{} & $O(n)$ \\
        cover tree \cite{} & $O(n)$ & $O(\krnum^8\log n)$ & $O(\krnum^{12}\log n)$ \\
        simplified cover tree & $O(n)$ & $O(\doubnum{}\log \aspect{})$ \\
                              &        & $O(\doubnum{}\log n)$ \\
        \bottomrule
    \end{tabular}
    %\end{tabularx}
    \caption{
        Summary of the runtime and space usage of several nearest neighbor data structures.
        Here $n$ represents the size of the dataset.
    }
\end{table}

%%%%%%%%%%%%%%%%%%%%%%%%%%%%%%%%%%%%%%%%%%%%%%%%%%%%%%%%%%%%%%%%%%%%%%%%%%%%%%%%

\section{Review of Metric Spaces in Machine Learning}

%%%%%%%%%%%%%%%%%%%%%%%%%%%%%%%%%%%%%%%%

\subsection{Examples of Metric Spaces}

\begin{example}
    Any subset of a metric space is also a metric space.
\end{example}

%%%%%%%%%%%%%%%%%%%%%%%%%%%%%%%%%%%%%%%%

\subsection{What is the dimension of a metric space?}

In this section we introduce two 

\begin{definition}
%A ball of radius $\radius$ in a metric space $\set X$ is defined to be
    We let $B_\set X(x,\radius)$ denote the ball centered around $x$ of radius $r$ in metric space $\set X$.
    That is,
\begin{equation}
B_\set X(x,\radius) = \{ y : y\in\set X, \dist{x}{y} \le \radius \}.
\end{equation}
\end{definition}

\begin{definition}
    Let $\set X$ be a metric space, and let $\mu : \{\set X\} \to \R^+$ be a measure on $\set X$.
    Then the \emph{expansion constant} is defined as
    \begin{equation}
        c_\set X = \max_{x\in\set X, \radius\in\R^+} \frac{\mu B_{\set X}(x,2\radius)}{\mu B_{\set X}(x,\radius)}
        ,
    \end{equation}
    and the \emph{expansion dimension} is defined as
    \begin{equation}
        \krdim\set X = \log_2 c_\set X
        .
    \end{equation}
\end{definition}

%%%%%%%%%%%%%%%%%%%%%%%%%%%%%%%%%%%%%%%%

\begin{definition}
    A $\delta$-covering of a metric space $\set X$ is a set $\{x_1,x_2,...,x_n\} \subseteq \set X$ such that for all $x\in\set X$, there exists an $x_i$ such that $\dist{x}{x_i} < \delta$.
    The $\delta$-covering number $N_\delta(\set X)$ is the cardinality of the smallest $\delta$-covering.
    The log of the covering number $\log N_\delta(\set X)$ is called the metric entropy of $\set X$.
\end{definition}

\begin{definition}
A $\delta$-packing of a metric space $\set X$ is a set $\{x_1,x_2,...,x_M\} \subseteq \set X$ such that $\dist{x_i}{x_j} > \delta$ for all distinct $i,j\in[M]$.
The $\delta$-packing number $M_\delta (\set X)$ is the cardinality of the largest $\delta$-packing.
\end{definition}

\begin{lemma}
    \label{lemma:coverpacking}
    For any metric space $\set X$ and any $\delta>0$,
    \begin{equation}
        M_{2\delta}(\set X) \le N_\delta(\set X) \le M_{\delta}(\set X)
        .
    \end{equation}
\end{lemma}

\begin{proof}
    To prove the first inequality, let $P$ be a $2\delta$-packing and $C$ be a $\delta$-cover of $\set X$.
    For every point $p\in P$, there must exist a $c\in C$ such that $\dist{p}{c}\le\delta$.
    No other $p'\in P$ can also satisfy $\dist{p'}{c}\le\delta$, because then by the triangle inequality
    \begin{equation}
        \dist{p'}{p} \le \dist{p'}{c}+\dist{p}{c} \le 2\delta
        ,
    \end{equation}
    which would contradict that $P$ is a $2\delta$-packing.
    In other words, for each $c\in C$, there is at most one $p\in P$.
    So $N_\delta \ge |C| \ge |P| \ge M_{2\delta}$.

    To prove the second inequality, let $\set X'\subseteq \set X$ be a maximal $\delta$-packing.
    Then there does not exist an $x\in\set X$ such that for all $x'\in\set X'$, 
    $\dist{x}{x'} > \delta$.
    (Otherwise, $\set X' \cup \{x\}$ would be a packing larger than $\set X'$.)
    Hence, $\set X'$ is also a $\delta$-cover,
    and the smallest $\delta$-cover can be no larger.
\end{proof}

%\cite{nickl2007bracketing} uses metric entropy to prove a version of the central limit theorem.
%
%\begin{example}
%\end{example}

%\begin{definition}
    %The Minkowski dimension of a metric space is defined to be
    %\begin{equation}
        %\minkdim \set X = \lim_{\delta\to0} \frac{\log N_\delta(\set X)}{\log 1/\delta}
        %.
    %\end{equation}
%\end{definition}
%
%\begin{example}
    %Let $\set X$ be a finite metric space.
    %Then $\minkdim \set X = 0$.
%\end{example}

\begin{definition}
    The \emph{doubling number} of a metric space is
    \begin{equation}
        \doubnum(\set X) = \max_{x\in\set X, \radius\in\R^+} N_\radius(B_{\set X}(x,2\radius))
        .
    \end{equation}
    The \emph{doubling dimension} of a metric space $\set X$ is the log of the doubling number.
    Specifically,
    \begin{equation}
        \doubdim \set X = \log \doubnum\set X = \max_{x\in\set X, \radius\in\R^+} \log N_\radius(B_{\set X}(x,2\radius))
        .
    \end{equation}
\end{definition}
\cite{gupta2003bounded}

\begin{lemma}[\cite{krauthgamer2004navigating},\cite{gupta2003bounded}]
    Every finite metric $(\set X,d)$ satisfies
    \begin{equation}
        \doubdim\set X \le 4\cdot\krdim\set X
        .
    \end{equation}
\end{lemma}
%\begin{proof}
    %Let $k=\krdim(\set X)$.
    %Fix some ball $B(x,2r)$.
    %We will show that $B(x,2r)$ can be covered by $k^4$ balls of radius $r$.
%\end{proof}

\begin{definition}
    Let $\set X$ be a finite metric space.
    The \emph{diameter} of $\set X$, denoted by $\diam{\set X}$, is the maximum distance between any two points.
    The \emph{separation} of $\set X$, denoted by $\codiam{\set X}$, is the minimum distance between any two points.
    The \emph{aspect ratio} of $\set X$, denoted by $\aspect{\set X}$, is the ratio of the diameter to the dispersion.
\end{definition}

\cite{bartal2003metric} discusses the relation between aspect ratios and tree metrics.

\begin{example}
    This example shows that there is not necessarily a relationship between the aspect ratio of a metric and either its expansion or doubling constants.
    Let $\set Y=\{y_1,...,y_n\}$ be the discrete metric space of size $n$;
    that is,
    \begin{equation}
        \dist{y_i}{y_j}=
        \begin{cases}
            0 & i = j \\
            1 & \text{otherwise}
        \end{cases}
        .
    \end{equation}
    Then the aspect ratio of $\set Y$ is 1 (i.e. as small as possible),
    and both the expansion and doubling constants of $\set Y$ are $n-1$ (i.e. arbitrarily large).
    Now construct the set $\set Y'=\{y'_1, y'_2, y'_3\}$.
    Let $r>2$, and define the distance function to be
    \begin{equation}
        d(y'_i,y'_j) =
        \begin{cases}
            0 & i=j \\
            1 & i=1, j=2 \\
            r & i=1, j=3 \\
            r & i=2, j=3 \\
        \end{cases}
        .
    \end{equation}
    Then the aspect ratio is $r$ (i.e. arbitrarily large),
    but the expansion constant is always 2
    and the doubling constant always 1.
\end{example}

\begin{lemma}[\cite{krauthgamer2004navigating}]
    For any metric space $\set X$, we have that
    %\begin{equation}
    $
        |\set X| \le \aspect{\set X}^{O(\doubdim{\set X})}.
    $
    %\end{equation}
\end{lemma}
\begin{proof}
    %Assume wlog that the separation distance is 1
    %(we can rescale all the distances to ensure this).
    %Then the diameter of $\set X$ is $\aspect{\set X}$.
\end{proof}

%%%%%%%%%%%%%%%%%%%%%%%%%%%%%%%%%%%%%%%%

\subsection{The growth rate of the aspect ratio}

The aspect ratio and the expansion number share the unattractive property that adding a single point to a dataset can increase these quantities arbitrarily.
Under mild assumptions, however, we can show that this is unlikely to happen.
Specifically, let $\set X$ be a metric space, 
and let $X=\{x_1,...,x_n\}\subset\set X$ be a sample of $n$ i.i.d.\ points from a distribution $\distribution D$ over $\set X$.
Our goal is to show that the aspect ratio is polynomial in $n$.
We will later show that the log of the aspect ratio bounds the depth of the cover tree (see Theorem \ref{}),
and so the depth of the cover tree will be logarithmic in $n$.

We begin by bounding the diameter of $X$.
%In general, it does not make sense to take the expectation with respect to $\distribution D$.
We say the distribution $\distribution D$ has \emph{finite expected distance} if there exists an $\bar x\in\set X$ such that $\mu=\E\dist{\bar x}{x_i}$ is finite.
Note that this is a mild condition satisfied by most standard distributions on Euclidean space.
For example, the uniform, Gaussian, exponential, Weibull, and Pareto distributions all have finite expected distance.
Notice that the Weibull and Pareto distributions have heavy tails.
One easy to describe distribution which does not satisfy this property is the Cauchy distribution 
(the distribution of the reciprocal of a Gaussian random variable).
The following lemma shows that this is a sufficient condition for the diameter to grow polynomial in $n$.

\begin{lemma}
    \label{lemma:Ediam}
    %Let $\set X$ be a metric space.
    %Let $X=\{x_1,...,x_n\}\subset\set X$ be a sample of $n$ i.i.d.\ points satisfying the following property:
    %There exists a $\bar x\in\set X$ such that $\mu=\E\dist{\bar x}{x_i}$ is finite.
    %Then, 
    %\begin{equation}
        %\E\diam{X} \le 2n\mu
    %\end{equation}
    Let $\set X$ and $X$ be defined as above,
    and assume that $\distribution D$ has finite expected distance.
    Then, $\E\diam{X} \le 2n\mu$.
\end{lemma}

\begin{proof}
    By the triangle inequality, we have that
    \begin{equation*}
        \diam{X}
        = 
        \max_{i,j} \dist{x_i}{x_j}
        \le
        \max_{i,j} (\dist{\bar x}{x_i} + \dist{\bar x}{x_j})
        %\\ &=
        %\max_i \dist{\bar x}{x_i} + \max_j\dist{\bar x}{x_j}
        =
        2\max_i \dist{\bar x}{x_i}
        .
    \end{equation*}
    We now remove the max using the union bound.
    This gives
    \begin{equation*}
        \prob{\diam{X} > t}
        \le
        \prob{\max_i 2\dist{\bar x}{x_i} > t}
        \le
        \sum_{i=1}^n\prob{2\dist{\bar x}{x_i} > t}
        \label{eq:Ediamub}
        =
        n\prob{2\dist{\bar x}{x_1} > t}
        %\label{eq:Ediamiid}
        .
    \end{equation*}
    %Equation \eqref{eq:Ediamub} follows from the union bound, 
    %and \eqref{eq:Ediamiid} follows because the $x_i$s are i.i.d.
    The rightmost equality follows because the $x_i$s are i.i.d.
    Finally, since the distances are always nonnegative, we have that
    \begin{equation*}
        \E\diam{X} 
        = 
        \int_0^\infty \prob{\diam{X} > t} \dd t
        \le
        \int_0^\infty n\prob{2\dist{\bar x}{x_1} > t} \dd t
        =
        %n\int_0^\infty \prob{2\dist{\bar x}{x_1} > t} \dd t
        %\\ &=
        2n \E\dist{\bar x}{x_1}
        %\label{eq:Ediamproof}
        .
    \end{equation*}
\end{proof}

%%%%%%%%%%%%%%%%%%%%%%%%%%%%%%%%%%%%%%%%

Next we show that the codiameter cannot shrink too fast.
We say that the distribution $\distribution D$ has \emph{$B$-bounded density} if
for all $x\in\set X$, the density of $\dist{x}{x_i}$ is bounded by $B$.
An immediate consequence is that 
\begin{equation}
    \max_{x\in\set X} \prob{\dist{x}{x_i} \le t} \le Bt
    .
\end{equation}
Again, all the standard distributions in Euclidean space satisfy this condition.
The following lemma shows that this condition is sufficient to lower bound the codiameter.

\begin{lemma}
    \label{lemma:Ecodiam}
    Let $\set X$ and $X$ be defined as above,
    and assume that $\distribution D$ has $B$-bounded density.
    Then, $\E\codiam{X} \ge (2n^2B)^{-1}$.
\end{lemma}
\begin{proof}
    We have that
    \begin{align}
        \prob{\codiam{X} \le t}
        &=
        \prob{\min \{ \dist{x_i}{x_j} : i\in\{1,...,n\}, j\in\{i+1,...,n\} \} \le t}
        %\prob{\min_{i\ne j} \dist{x_i}{x_j} \le t}
        \\ &\le 
        \sum_{i=1}^n\sum_{j=i+1}^n \prob{\dist{x_i}{x_j} \le t}
        \label{eq:lemcodiam1}
        %\\ &\le
        %\sum_{i=1}^n n \max_{x\in X} \prob{\dist{x_i}{x} \le t}
        \\ & \le
        n^2 \max_{x\in X} \prob{\dist{x_1}{x} \le t}
        \label{eq:lemcodiam2}
        \\ & \le 
        n^2 B t
        \label{eq:lemcodiam3}
    \end{align}
    Equation \eqref{eq:lemcodiam1} follows from the union bound,
    \eqref{eq:lemcodiam2} from the fact that the $x_i$s are i.i.d.,
    and \eqref{eq:lemcodiam3} from the definition of $B$-bounded.
    We further have that since probabilities are always no greater than 1,
    \begin{equation}
        \prob{\codiam{X}\le t} \le \min\{1,n^2Bt\}
        .
    \end{equation}
    Finally, since $\codiam{X}$ is nonnegative, we have that 
    \begin{align}
        \E \codiam{X}
        &=
        \int_0^\infty (1-\prob{\codiam{X} \le t}) \dd t
        %\\ & =
        %\int_0^\infty \prob{\codiam{X} > t} dt
        \\ & \ge
        \int_0^\infty (1-\min\{1,n^2 Bt\}) \dd t
        \\ & = 
        \int_0^{(n^2B)^{-1}} (1 - n^2 Bt) \dd t
        \\ & =
        %\frac{1}{n^2B} - \frac{\left(\frac{1}{n^2B}\right)^3}{2}
        %\\ & \ge
        \frac{1}{2n^2B}
        \label{eq:Ecodiamproof}
        .
    \end{align}
\end{proof}

%%%%%%%%%%%%%%%%%%%%%%%%%%%%%%%%%%%%%%%%

An immediate consequence of Lemmas \ref{lemma:Ediam} and \ref{lemma:Ecodiam} is the following bound on the aspect ratio.

\begin{lemma}
    \label{lemma:Easpect}
    Let $\set X$ and $X$ be defined as above.
    Assume that $\distribution D$ has finite expected distance and $B$-bounded density.
    Then, $\E\aspect{X} \le2B\mu n^3$. 
\end{lemma}

%%%%%%%%%%%%%%%%%%%%%%%%%%%%%%%%%%%%%%%%%%%%%%%%%%%%%%%%%%%%%%%%%%%%%%%%%%%%%%%%

\section{The Simplified Cover Tree}

A \emph{simplified cover tree} is a data structure for efficiently representing a finite metric space.
Each node in the tree corresponds to exactly one point in the space,
and the tree obeys the following three invariants.
\begin{enumerate}
    \item \emph{Leveling invariant}.
    Every node $\p$ has an associated integer $\level\p$.
    For all nodes $\q\in\children\p$, $\level\q < \level\p$.
    (Note that $\level{p}$ can be negative or arbitrarily large.
    In particular, it is not the depth of $p$ in the tree.)
    \item \emph{Covering invariant}.
    Every node $\p$ has an associated real number $\covdist\p=2^{\level\p}$.
    For all nodes $\q\in\children\p$, $\dist \p \q \le \covdist\p$.%
    \footnote{
        As in the original cover tree, practical performance is improved on most datasets by redefining $\covdist p = 1.3 ^ {\level p}$.
        All of our experiments use this modified definition.
    }
    \item \emph{Separating invariant}.
    For all nodes $\q_1,\q_2\in\children\p$, $\dist {\q_1} {\q_2} \ge \covdist\p/2$.
\end{enumerate}
It will be useful to define the function
\begin{equation}
\maxdist p = \argmax_{q\in\descendants{p}} \dist p q
.
\end{equation}
In words, $\maxdist\p$ is the greatest distance from $p$ to any of its descendants.
This value is upper bounded by $2^{\level{p}+1}$, 
and its exact value can be cached within the data structure.

\begin{remark}
    The original version of the simplified cover tree \citep{izbicki2015faster} had a slightly different leveling invariant.
    The original version required that the level of a child be exactly one less than the level of the parent.
    The version used in this thesis is strictly more general and facilitates the runtime analysis of the tree.
\end{remark}

%%%%%%%%%%%%%%%%%%%%%%%%%%%%%%%%%%%%%%%%

\subsection{Properties of the Simplified Cover Tree}

Before we present algorithms for manipulating the cover tree, 
we present two lemmas that bound the shape of the cover tree.
These lemmas are a direct consequence of the cover tree's invariants and motivate the invariants' selection.
This section can be safely skipped by the reader not interested in the details of the tree's runtime analysis.

\begin{lemma}
    \label{lemma:children}
    For every node $p$ in a cover tree, we have that
    $|\children\p| \le \doubnum^2$.
\end{lemma}

\begin{proof}
    To simplify notation, we let $\delta=\covdist{p}$.
    %The covering invariant ensures that all the children of $p$ are contained in $B(p,\delta)$,
    The separating invariant ensures that the children of $p$ form a $\delta/2$-packing of $B(p,\delta)$.
    So by the definition of $M$ and Lemma \ref{lemma:coverpacking}, we have
    \begin{equation}
        |\children{p}| 
        \le M_{\delta/2}(B(p,\delta)) 
        \le N_{\delta/4}(B(p,\delta)) 
        %\le \doubnum N_{\delta/2}(B(p,\delta)) 
        %\le \doubnum^2
        .
    \end{equation}
    We now show that $N_{\delta/4}(B(p,\delta))\le\doubnum$.
    Let $Y$ be a $\delta/2$-covering of $B(p,\delta)$.
    For each $y_i\in Y$, let $Y_i$ be a minimum $\delta/4$-covering of $B(y_i,\delta/2)$.
    The union of the $Y_i$s is a $\delta/4$-covering of $B(p,\delta)$.
    There are at most $\doubnum$ $Y_i$s, and each $Y_i$ contains at most $\doubnum$ elements.
    So their union contains at most $\doubnum^2$ elements.
\end{proof}

\begin{lemma}
    \label{lemma:height}
    Let $\height{p}$ denote the number of edges between $p$ and its most distant leaf node.
    We have that $\height{p} \le \log_2\aspect{X}$.
\end{lemma}

\begin{proof}
    Define the recursive family of sets 
    \begin{equation}
        Y_i = 
        \begin{cases}
            \{p\} & i=0
            \\
            \bigcup\limits_{y\in Y_{i-1}} \children{y} & i >0
        \end{cases}
    \end{equation}
    In words, $Y_i$ contains all the data points from the uppermost $i$ levels of the tree,
    and in particular $Y_{\height p}=X$.
    Due to the covering invariant, we have that $\codiam{Y_1} \le \covdist{p}/2 \le \diam{X}/2$.
    By induction, it holds that $\codiam{Y_i} \le \diam{X}/2^i$,
    and so $\codiam{X} \le \diam{X}/2^{\height p}$.
    Substituting into the definition of the aspect ratio gives
    \begin{equation}
        \aspect{X} 
        = \frac{\diam{X}}{\codiam{X}} 
        \ge \frac{\diam{X}}{\diam{X}/2^{\height p}} 
        = 2^{\height p}
        .
    \end{equation}
    Solving for $\height p$ gives the desired result.
\end{proof}

%%%%%%%%%%%%%%%%%%%%%%%%%%%%%%%%%%%%%%%%%%%%%%%%%%%%%%%%%%%%%%%%%%%%%%%%%%%%%%%%

\subsection{Algorithms}

\begin{algorithm}[H]
\caption{Simplified cover tree insertion}
\label{alg:insert}

    \vspace{0.1in}
{\bfseries function} \ctinsert(cover tree $p$, data point $x$)

\begin{algorithmic}[1]
    \If {$\dist p x > \covdist p$}
        \State Create a new node $q$
        \State $dp(q) \leftarrow x $
        \State $\level q \leftarrow \ceil{\log_2 d(p,q)}$
        \State $\children{q} \leftarrow \{p\}$
        \State\Return $q$
    \Else
        \For {$q \in \children{p}$}
            \If {$\dist q x \le \covdist q$}
                \State $q' \leftarrow \ctinsert(q,x)$
                \State $p' \leftarrow p$ with child $q$ replaced with $q'$
                \State \Return $p'$
            \EndIf
        \EndFor
        \State // x was not added to any child
        \State\Return $p$ with $x$ added as a child
    \EndIf
\end{algorithmic}
\end{algorithm}

\begin{theorem}
    The runtime of $\ctinsert$ is $O(\doubnum^2\log_2\aspect{X})$.
\end{theorem}
\begin{proof}
    If $\dist{p}{x} > \covdist{p}$, then the running time is constant.
    Otherwise, the algorithm loops over all the children of $p$ and recurses on at most one child.
    There at most $\doubnum^2$ children (Lemma \ref{lemma:children}),
    and the height of the tree is at most $\log_2\aspect{X}$ (Lemma \ref{lemma:height}).
\end{proof}

%%%%%%%%%%%%%%%%%%%%%%%%%%%%%%%%%%%%%%%%

\begin{algorithm}[H]
    \caption{Merging cover trees}
    \label{alg:merge}
    \vspace{0.1in}
    {\bfseries function} \ctmerge(cover tree $p$, cover tree $q$)

\begin{algorithmic}[1]
\end{algorithmic}
\end{algorithm}

%%%%%%%%%%%%%%%%%%%%%%%%%%%%%%%%%%%%%%%%


\begin{algorithm}[H]
    \caption{Simplified cover tree nearest neighbor query}
    \label{alg:query}

    \vspace{0.1in}
{\bfseries function} \findnn(cover tree $p$, query  point $x$, nearest neighbor so far $y$)

\begin{algorithmic}[1]
    \If {$d(p,x) < d(y,x)$}
        \State $y \leftarrow p$
    \EndIf
    \For {each child $q$ of $p$ sorted by distance to $x$}
        \If {$d(y,x) > d(y,q) - \maxdist{q}$} %2^{\level q}$}
            \State $y \leftarrow \findnn(q,x,y)$
        \EndIf
    \EndFor
    \State\Return $y$
\end{algorithmic}
\end{algorithm}

%%%%%%%%%%%%%%%%%%%%%%%%%%%%%%%%%%%%%%%%

%%%%%%%%%%%%%%%%%%%%%%%%%%%%%%%%%%%%%%%%%%%%%%%%%%%%%%%%%%%%%%%%%%%%%%%%%%%%%%%%

\section{Analysis}

\end{document}
