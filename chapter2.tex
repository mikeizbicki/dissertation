\documentclass[thesis.tex]{subfiles}

\newcommand{\set}[1]{\mathcal {#1}}
\newcommand{\sized}[1]{\tilde \set {#1}}
\newcommand{\mappend}[1]{\oplus_{\set {#1}}}
\newcommand{\mempty}[1]{\epsilon_{\set {#1}}}
\newcommand{\monoid}[1]{(\set {#1}, \mappend {#1}, \mempty {#1})}

\begin{document}
\chapter{Chapter 2}

From a certain perspective, everything in this dissertation could be described as ``obvious.''
One of my main contributions is to formalize the obvious.
This gives the practitioner a way to easily and systematically derive new algorithms.
Despite all of these techniques being obvious,
as we shall see,
they are not widely used in practice or research.

\begin{definition}
    A \emph{monoid} is a tuple $(\set A, \mappend A, \mempty A)$,
    where $\set A$ is a set,
    $\mappend A : \set A \times \set A \to \set A$ is a binary operation,
    and $\mempty A \in \set A$ if it obeys the following two laws:
    \begin{align}
    \tag{associativity}
    \forall a_1, a_2, a_3 \in \set A. & (a_1 \mappend A a_2) \mappend A a_3 &= a_1 \mappend A (a_2 \mappend A a_3)
    \\
    \tag{identity}
    \end{align}
\end{definition}

\begin{definition}
    A \emph{homomorphism} between two monoids $\monoid A$ and $\monoid B$ is a function  $f: \set A \to \set B$ satisfying
    \begin{equation}
        f(a_1 \mappend A a_2) = f(a_1) \mappend B f(a_2)
    \end{equation}
\end{definition}

\begin{definition}
    A \emph{learning algorithm} is a function from $\{\set Z\} \to \set W$.
    We call a learning algorithm \emph{homomorphic} if there exists an operation $\mappend W$ such that $\monoid W$ is a monoid and $A$ is a homomorphism.
\end{definition}

%\begin{remark}
    %For any monoidal parameter space $\monoid W$, we can define the sized monoid %$\monoid {\sized W}$ as follows.
%\end{remark}

\begin{example}
    Sized monoid
\end{example}

\begin{example}
    product monoid
\end{example}

\begin{example}
    trivial learner, bad runtime properties
\end{example}

\section{Examples}

\subsection{Exponential Family}

\subsection{Moment Estimators}

\subsection{Empirical Characteristic function}
\cite{yu2004empirical}

\subsection{Ridge Regression}

\subsection{The Free Monoid}

\subsection{High Dimensional Logistic Regression}

\subsection{Information Theoretic Metric Learning}

\subsection{Averaging}

\subsection{Cupulae}

\subsection{One step estimators}

\subsection{MCMC}

\end{document}
