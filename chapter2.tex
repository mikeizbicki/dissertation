\documentclass[thesis.tex]{subfiles}

\newcommand{\set}[1]{\mathcal {#1}}
\newcommand{\sized}[1]{\tilde \set {#1}}
\newcommand{\mappend}[1]{\oplus_{\set {#1}}}
\newcommand{\mempty}[1]{\epsilon_{\set {#1}}}
\newcommand{\monoid}[1]{(\set {#1}, \mappend {#1}, \mempty {#1})}

\newcommand{\mse}[1]{}

\newcommand{\eye}[1]{I_{#1}}
\newcommand{\p}[1]{p({#1})}
\newcommand{\psup}[2]{\hat p^{#1}({#2})}
\newcommand{\phat}[1]{\psup{}{#1}}
\newcommand{\pkde}[1]{\psup{\text{kde}}{#1}}
\newcommand{\pnp}[1]{\psup{\text{NWX,nonparametric}}{#1}}
\newcommand{\psp}[1]{\psup{\text{NWX,semiparametric}}{#1}}
\newcommand{\gaussian}[3]{\mathcal N({#1};{#2},{#3})}

%%%%%%%%%%%%%%%%%%%%%%%%%%%%%%%%%%%%%%%%%%%%%%%%%%%%%%%%%%%%%%%%%%%%%%%%%%%%%%%%

\begin{document}
\chapter{Chapter 2}

From a certain perspective, everything in this dissertation could be described as ``obvious.''
One of my main contributions is to formalize the obvious.
This gives the practitioner a way to easily and systematically derive new algorithms.
Despite all of these techniques being obvious,
as we shall see,
they are not widely used in practice or research.

\begin{definition}
    A \emph{monoid} is a tuple $(\set A, \mappend A, \mempty A)$,
    where $\set A$ is a set,
    $\mappend A : \set A \times \set A \to \set A$ is a binary operation,
    and $\mempty A \in \set A$ if it obeys the following two laws:
    \begin{align}
    \tag{associativity}
    \forall a_1, a_2, a_3 \in \set A. & (a_1 \mappend A a_2) \mappend A a_3 &= a_1 \mappend A (a_2 \mappend A a_3)
    \\
    \tag{identity}
    \end{align}
\end{definition}

\begin{definition}
    A \emph{homomorphism} between two monoids $\monoid A$ and $\monoid B$ is a function  $f: \set A \to \set B$ satisfying
    \begin{equation}
        f(a_1 \mappend A a_2) = f(a_1) \mappend B f(a_2)
    \end{equation}
\end{definition}

\begin{definition}
    A \emph{learning algorithm} is a function from $\{\set Z\} \to \set W$.
    We call a learning algorithm \emph{homomorphic} if there exists an operation $\mappend W$ such that $\monoid W$ is a monoid and $A$ is a homomorphism.
\end{definition}

%\begin{remark}
    %For any monoidal parameter space $\monoid W$, we can define the sized monoid %$\monoid {\sized W}$ as follows.
%\end{remark}

\begin{example}
    Sized monoid
\end{example}

\begin{example}
    product monoid
\end{example}

\begin{example}
    trivial learner, bad runtime properties
\end{example}

%%%%%%%%%%%%%%%%%%%%%%%%%%%%%%%%%%%%%%%%%%%%%%%%%%%%%%%%%%%%%%%%%%%%%%%%%%%%%%%%

\section{Examples: Sufficient Statistics}

\subsection{Exponential Family}

\subsection{Ridge Regression}

\subsection{Empirical Characteristic function}
\cite{yu2004empirical}

\subsection{High Dimensional Logistic Regression}

\subsection{Information Theoretic Metric Learning}

%%%%%%%%%%%%%%%%%%%%%%%%%%%%%%%%%%%%%%%%%%%%%%%%%%%%%%%%%%%%%%%%%%%%%%%%%%%%%%%%

\section{Examples: Optimization}

\subsection{Averaging}

\subsection{One step estimators}

%\section{Examples}
%
%\subsection{The Free Monoid}
%
%\subsection{Moment Estimators}
%
%\subsection{Cupulae}

%%%%%%%%%%%%%%%%%%%%%%%%%%%%%%%%%%%%%%%%%%%%%%%%%%%%%%%%%%%%%%%%%%%%%%%%%%%%%%%%

\section{Examples: Bayesian Methods}

Bayesian machine learning methods estimate the \emph{posterior distribution} $\p{\theta|Z}$ of a parameter $\theta$ given data set $Z$.

%%%%%%%%%%%%%%%%%%%%%%%%%%%%%%%%%%%%%%%%

\subsection{Gaussian posteriors}

The simplest method of merging posterior distributions was proposed by \cite{neiswanger2013asymptotically}.
They assume that the posterior distribution is approximately normal.
That is,
\begin{equation}
    \p{\theta | Z_i}
    \approx
    \gaussian{\theta}{\hat\mu_i}{\hat\Sigma_i}
    .
\end{equation}
This assumption is justified by the Bernstein-von Mises theorem.%
\footnote{
    The Bernstein-von Mises theorem is also often called the Bayesian central limit theorem.
}

The final posterior density is then given by
\begin{equation}
    \phat{\theta | Z}
    =
    \prod_{i=1}^m
    \phat{\theta | Z_i}
    \approx
    \gaussian{\theta}{\hat\mu}{\hat\Sigma}
    ,
\end{equation}
where
\begin{align}
    \hat\Sigma
    &=
    \left(
        \sum_{i=1}^m
        \hat\Sigma_i^{-1}
    \right)^{-1}
    ,
    \\
    \hat\mu
    &=
    \hat\Sigma \left(\sum_{i=1}^m \hat\Sigma^{-1} \mu_i \right)
    .
\end{align}

%%%%%%%%%%%%%%%%%%%%%%%%%%%%%%%%%%%%%%%%

\subsection{Nonparametric and semiparametric merging}

Recent work has focused on improving non-asymptotic bounds of MCMC 
\citep[e.g.][]{latuszynski2013nonasymptotic,andrieu2015convergence,adamczak2015exponential,andrieu2016establishing}.

\cite{neiswanger2014asymptotically} also present asymptotically efficient estimators.
We can write the kernel density estimate of the posterior as
\begin{equation}
    \pkde{\theta|Z}
    =
    \frac{1}{Z}\sum_{z\in Z}
    \frac{1}{h}k(\theta,z)
    =
    \frac{1}{Z}\sum_{z\in Z}
    \gaussian{\theta}{z}{h^2 \eye d}
    ,
\end{equation}
where $\eye d$ is the $d$ dimensional identity matrix.
\begin{align}
    \pnp{\theta|Z}
    &=
    \prod_{i=1}^m \pkde{\theta|Z_i}
    \\
    &=
    \frac{1}{n^m}
    \prod_{i=1}^m 
    \sum_{z\in Z_i}
    \gaussian{\theta}{z}{h^2 \eye d}
    \\
    &\propto
    \sum_{z_1\in Z_1}
    \dots
    \sum_{z_m\in Z_m}
    \gaussian{\theta}{z_{\{z_1,...,z_m\}}}{h^2 \eye d}
\end{align}
where
\begin{align}
    \bar\theta_{\{z_1,...,z_m\}} 
    &= 
    \frac{1}{m}\sum_{z_i\in\{z_1,...,z_m\}} z_i
    ,
    \\
    w_{\{z_1,...,z_m\}} 
    &=
    \prod_{i=1}^m\gaussian{z_i}{\bar\theta_{\{z_1,...,z_m\}}}{h^2\eye d}
\end{align}

\begin{theorem}
    If $h \asymp T^{-1/d}$,
    then
    \begin{equation}
        \mse{}
        \E((\phat{\theta} - \p{\theta})^2)
    \end{equation}
\end{theorem}

\end{document}
