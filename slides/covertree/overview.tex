\newcommand{\cexp}{c_\textnormal{exp}}
\newcommand{\cdoub}{c_\textnormal{doub}}
\newcommand{\cdoubstar}{c_{\textnormal{doub}^*}}
\newcommand{\chole}{c_\textnormal{hole}}

%\newcommand{\poly}[1]{\text{poly}(#1)}
\newcommand{\poly}[1]{#1^{O(1)}}


\newcommand{\nn}[1]{\ensuremath{\ensuremath{{{#1}}_{nn}}}}
\newcommand{\dist}[2]{\ensuremath{\ensuremath{d}({{#1}},{{#2}})}}
\newcommand{\exprad}[1]{\ensuremath{\ensuremath{2}}}
\newcommand{\pack}{\ensuremath{\text{\ttfamily pack}}}
\newcommand{\rmNodes}{\ensuremath{\text{\ttfamily rmNodes}}}
\newcommand{\findnn}{\ensuremath{\text{\ttfamily findNearestNeighbor}}}
\newcommand{\ctmerge}{\ensuremath{\text{\ttfamily merge}}}
\newcommand{\ctinsert}{\ensuremath{\text{\ttfamily insert}}}
\newcommand{\ctinsertHelper}{\ensuremath{\text{\ttfamily insert\_}}}
\newcommand{\rebalance}{\ensuremath{\text{\ttfamily rebalance}}}
\newcommand{\rebalanceHelper}{\ensuremath{\text{\ttfamily rebalance\_}}}
\newcommand{\mkfunction}[1]{\ensuremath{\text{\ttfamily {#1}}}}
\newcommand{\mkvar}[1]{\ensuremath{\text{\emph{{#1}}}}}
\newcommand{\nullvar}{\ensuremath{\text{\ttfamily null}}}
\newcommand{\datapoint}[1]{\ensuremath{\text{\ttfamily dp}({#1})}}
\newcommand{\level}[1]{\ensuremath{\text{\ttfamily level}({#1})}}
\newcommand{\sepdist}[1]{\ensuremath{\text{\ttfamily sepdist}({#1})}}
\newcommand{\covdist}[1]{\ensuremath{\text{\ttfamily covdist}({#1})}}
\newcommand{\children}[1]{\ensuremath{\text{\ttfamily children}({#1})}}
\newcommand{\descendants}[1]{\ensuremath{\text{\ttfamily descendants}({#1})}}
\newcommand{\maxdist}[1]{\ensuremath{\text{\ttfamily maxdist}({#1})}}

%%%%%%%%%%%%%%%%%%%%%%%%%%%%%%%%%%%%%%%%%%%%%%%%%%%%%%%%%%%%%%%%%%%%%%%%%%%%%%%%

\begin{frame}{Cover tree overview}

    %\Large

%Cover trees are a data structure for faster nearest neighbor queries in non-metric spaces.
    The cover tree data structure (Beygelzimer, et al. 2006) is used for:
\begin{itemize}
    \item $k$-nearest neighbor classification
    \item kernel support vector machines
    \item gaussian processes
    \item other non-parametric learning algorithms
    \item arbitrary metric spaces, not just Euclidean space
\end{itemize}

\vspace{0.15in}
Contibutions that I'll present
    \begin{itemize}
        \item simplified cover tree definition
        \item provided merge algorithm
        \item improved run time bounds
        \item new notion of intrinsic dimension
        \item experiments on large scale protein and image data
    \end{itemize}
%\vspace{0.1in}
%\hrule
\end{frame}

%%%%%%%%%%%%%%%%%%%%%%%%%%%%%%%%%%%%%%%%

\ignore{
\begin{frame}{Cover tree outline}
    \Large
\begin{itemize}
        \Large
    \item metric spaces
        \begin{itemize}
                \Large
            \item for protein data
            \item for image data
        \end{itemize}
        \vspace{0.1in}
    \item contibutions that I'll present
        \begin{itemize}
                \Large
            \item simplified cover tree definition
            \item provided merge algorithm
            \item improved run time bounds
            \item new notion of intrinsic dimension
            \item experiments on large scale protein and image data
        \end{itemize}
        \vspace{0.1in}
    \item other contributions

\end{itemize}

%\begin{itemize}
%
%\item
%Cover trees are a data structure for nearest neighbor queries in arbitrary metric spaces
%\begin{itemize}
%\item
%makes $k$-nearest neighbor classification faster
%\item
%other algorithms that can be sped up include SVMs, and kernelized algorithm, and dimensionality reduction
%\item
%non-Euclidean metrics provide state-of-the-art performance on many learning problems
%\end{itemize}
%
%\item
%My original paper, 
%\begin{itemize}
%\item 
%simplifies the definition of the cover tree
%
%\item
%introduces the nearest ancestor invariant
%
%\item
%provides a cache-efficient memory layout for the tree
%
%\item
%shows how to merge two cover trees together
%
%\end{itemize}
%
%\item
%These contributions made the cover tree faster in practice, but did not improve the theoretical guarantees
%
%\item
%My goal for the dissertation is to also improve the theoretical guarantees
%
%\end{itemize}

\end{frame}
}

