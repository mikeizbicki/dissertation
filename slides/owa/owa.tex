%%%%%%%%%%%%%%%%%%%%%%%%%%%%%%%%%%%%%%%%

\begin{frame}
\frametitle{The \uncover<1-6>{full} optimal weighted average (OWA)}

Procedure:
\uncover<2-7>{
\vspace{0.1in}
\begin{itemize}
\item
In the \textbf{map} phase,
calculate the $\wmle_i$ as before
}

\uncover<3-7>{
\vspace{0.1in}
\item
In the \textbf{reduce} phase,
%On the second round, 
\begin{itemize}
\item 
Let $\Wowa = \vecspan\{\wmle_i\}_{i=1}^m$\uncover<7>{, $Z^{owa}$ be a new dataset}
}

\uncover<4-7>{
\item Calculate
\begin{align}
\label{eq:afull}
\wowa{}^{\uncover<1-6>{,full}} &= \argmin_{\w\in\Wowa} \sum _{(\x,y)\in Z^{\textit{\uncover<7>{owa}}}} \loss\left(y,\trans\x\w \right)
+
\lambda \reg(\w)
\end{align}
}
\end{itemize}
\end{itemize} 

%\vspace{0.1in}
Graphical Intuition
\begin{center}
\begin{tikzpicture}
    [
    dot/.style = {minimum width=0.15cm,inner sep=0pt,line width=0pt,fill,circle,black,font=\small}
    , yscale=0.65
    ]
\small
\draw[gray] plot [smooth cycle,tension=1] coordinates {(-0.5,-0.75) (-1.05,1) (-0.1,-0.1) (0.75,-0.5) (0.45,-0.7) };
\draw[gray] plot [smooth cycle,tension=1] coordinates {(-0.85,-0.85) (-1.3,1.55) (0.25,0) (2.5,-0.5) (1,-0.95)};
\draw[gray] plot [smooth cycle,tension=1] coordinates {(-1.15,-1.2) (-1.5,2) (1,0) (5,0) (3,-0.95)};

\node[dot] (wstar) at (-0.28,-0.25) {};
\node at (0.05,-0.55) {$\wmle$};

\uncover<3-7>{
\draw[thick] (-2.2,2.15) -- (5.2,1.03);
\node at (5.65,1.0) {$\Wowa$};
}

\uncover<2-7>{
%\node[dot] (wstarproj) at (0.1,1.8) {};
%\draw (wstar) -- (wstarproj);
%\draw (0.07,1.65) -- (0.23,1.62) -- (0.26,1.78);

%\node[dot] (wavestar) at (4,0.5) {};
%\node                 at (4.5,0.3) {$\E\wave$};
\node[dot] (wave) at (2.80,1.40) {};
\node at (3.1,1.85) {$\wmle_1$};
\node[dot] (wave) at (4.75,1.10) {};
\node at (4.9,1.55) {$\wmle_2$};
%\node[dot] (wave) at (3.8,1.25) {};

}

\uncover<4-7>{
\node[dot] at (-1.35,2.03) {};
\node at (-1.0,2.5) {$\wowafull$};
}

\uncover<6-7>{
\node[dot] (wave) at (3.8,1.25) {};
\node at (4.0,1.7) {$\wave$};
}

\uncover<7>{
\node[dot] at (-0.55,1.9) {};
\node at (-0.1,2.3) {$\wowa$};
}

\node at (4,-1.3) {$\loss(y,\trans\x\w)+\lambda \reg(\w)$};
%\node at (6.95,1.0) {$\Wowa = \vecspan\{\wmle_i\}_{i=1}^m$};

\end{tikzpicture}
\end{center}

\end{frame}

%%%%%%%%%%%%%%%%%%%%%%%%%%%%%%%%%%%%%%%%

\begin{frame}{Efficiently calculating $\wowa$}

%Let $\matW = \bigg(\wmle_1, \wmle_2, ..., \wmle_m\bigg) \in \mathbb R^{d\times m}$
%
%\vspace{0.1in}
%Then every $\w\in\Wowa$ can be written as $\w=\matW\alpha$, where $\alpha\in\mathbb R^m$

\vspace{0.1in}
We can rewrite $\wowa$ as
%$\wowa = \matW \ahat$
%where
\begin{align}
\wowa &= \matW \ahat
\\
\label{eq:afull}
\ahat &= \argmax_{\alpha\in\mathbb R^m} \sum _{(\x,y)\in \Zowa} \loss\left(y,\trans\x \matW \alpha \right)
+
\lambda \reg(\matW\alpha)
\\
\matW &= \bigg(\wmle_1, \wmle_2, ..., \wmle_m\bigg) \in \mathbb R^{d\times m}
\end{align}

%In the second round of communication
%\begin{itemize}
%\item Each machine transmits $\trans\x\matW$ for each data point in $\Zowa$
%\item This is $O(m)$ bits per data point 
%
%~~~~~~~~~$O(m\nowa)$ bits per machine
%
%~~~~~~~~~$O(m^2\nowa)$ bits total
%%\item Whenever $m\nowa < d$, fewer bits are transfered than the first round
%\end{itemize}

\pause
Notice that:
\begin{itemize}
%\item
%The merging procedure depends on the data,
%so existing impossibility theorems do not apply.

%\pause
\item
The $\ahat$ optimization happens in a low dimensional subspace, so few datapoints are needed.

\pause
\item
The $\trans\x\matW$ in $(11)$ does not depend on $\alpha$, so needs to be computed only once (instead of each iteration).

%\pause
%\item
%The $\Zowa$ dataset should be stored already on the reduce server.
%
%(In the paper, we provide an algorithm for efficiently communicating the $\Zowa$ dataset to the reducer if needed.)

\end{itemize}

\end{frame}


