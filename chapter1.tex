\documentclass[../main.tex]{subfiles}
 
\begin{document}
\chapter{Introduction}


%\defn{Parallel learning algorithms} use multiple process

Many datasets are too large to fit in the memory of a single machine.
To analyze them, we must partition the data onto many machines and use distributed algorithms.
Existing distributed learning algorithms fall into one of two categories:

\defn{Interactive distributed algorithms} require many rounds of communication between machines.
Representative examples include \citet{boyd2011distributed}, \citet{li2014scaling}, \cite{ma2015adding}, and \cite{zhao2017scope}. 
These algorithms resemble standard iterative algorithms where each iteration is followed by a communication step. 
The appeal of interactive algorithms is that they enjoy the same statistical performance as standard sequential algorithms.
That is, given $m$ machines each with $n$ data points of dimension $d$, interactive algorithms have statistical error that decays as $O(\sqrt{d/mn})$.
But, interactive algorithms have three main disadvantages.
First, these algorithms are slow when communication latency is the bottleneck.
An extreme example occurs in the \emph{federated learning} environment proposed by \cite{mcmahan2017communication}, which uses cell phones as the computational nodes. 
Second, these algorithms require special implementations.
They do not work with off-the-shelf statistics libraries provided by (for example) Python, R, and Matlab.
Third, because of the many rounds of communication, any sensitive information in the data is likely to leak between machines.

\defn{Non-interactive distributed algorithms} require only a single round of communication.
Each machine independently solves the learning problem on a small subset of data,
then a master machine merges the solutions together.
These algorithms solve all the problems of interactive ones:
they are fast when communication is the main bottleneck;
they are easy to implement with off-the-shelf statistics packages;
and they are robust to privacy considerations.
The downside is worse statistical performance.
The popular averaging estimator has worst case performance $O(\sqrt{d/n})$ completely independent of the number of machines $m$. 
A growing body of work improves the analysis of the averaging estimator under special conditions 
(e.g.
\citet{mcdonald2009efficient},
\citet{zhang2012communication},
\citet{zhang2013divide},
and
\citet{rosenblatt2016optimality})
and develops more robust non-interactive estimators
(e.g.
\citet{zinkevich2010parallelized},
\citet{liu2014distributed},  
\citet{lee2015communication}, 
\citet{battey2015distributed},
\citet{han2016bootstrap},
and \citet{jordan2016communication}).
Existing estimators either work on only a limited class of models or have computationally intractable merge procedures.

\cite{karloff2010model} introduce the \defn{map reduce class} model of computation as an alternative to the PRAM model for analyzing algorithms.
Specifically, they define $MRC^i$ to be the class of functions computable on a MapReduce architecture with $O(\log^i n)$ rounds of computation.
Our model of computation is a subset of $MRC^0$,
however we require exactly one round of computation rather than a constant number of rounds.

\fixme{Differential privacy}

\end{document}


